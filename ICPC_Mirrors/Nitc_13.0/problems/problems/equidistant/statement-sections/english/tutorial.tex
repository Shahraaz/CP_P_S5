{
    \parindent=1cm
    \begin{tabular}{l@{\extracolsep{1cm}}l}
         Идея задачи: & Михаил Мирзаянов\\
         Условие: & Борис Минаев\\
         Проверяющая программа: & Борис Минаев\\
         Тесты: & Борис Минаев\\
     \end{tabular}
}

Подвесим дерево за вершину $c_1$ и найдём самую глубокую команду $c_f$ и соответствующую ей глубину $d$. Пусть $v$~--- вершина на середине пути от $c_1$ до $c_f$. Докажем, что либо она является ответом, либо ответа не существует.

Заметим, что ответ должен лежать в поддереве вершины $v$ ($T_v$), так как в противном случае расстояние до $c_1$ будет меньше расстояния до $c_f$.

Для всех команд не из $T_v$, расстояние до любой вершины из $T_v$ одинаковы. Таким образом, либо любая вершина из $T_v$ будет ответом для таких команд, либо ответа не существует.

С другой стороны, все команды из $T_v$ имеют глубину не большую $d$, иначе $c_f$ была бы не самой глубокой вершиной. Если все они имеют глубину ровно $d$, то $v$ является ответом для команд из $T_v$. В противном случае, для этих команд решение не существует.

Таким образом, всё что нам требуется~--- проверить, что вершина $v$ является ответом. Сделаем это, переподвесив дерево за неё и проверив, что глубины всех команд равны $d$.

Таким образом, задача решается за два обхода в глубину, а общее время работы составляет $O(n)$.