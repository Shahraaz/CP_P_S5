If it is not possible to change the state of array \t{\$a} from $s$ to $t$, output a single line containing a single integer $-1$. 

Otherwise, the first line of the output must contain an integer $k$~--- the number of lines in the program~($0 \le k \le 10\,000$). You don't have to minimize this value.

Each of the following $k$ lines should be either in the form

\begin{center} 
\t{foreach (\$a as~\&\$x) if (\$x == <some integer value>) break;}
\end{center}

or 

\begin{center} 
\t{foreach (\$a as~~\$x) if (\$x == <some integer value>) break;}
\end{center}

All integers should be positive and should not exceed 100. No other variables or language constructions are allowed. You should follow the format as close as possible, including whitespaces (e.~g. there should be two spaces between \t{as} and \t{\$x} in the non-reference form). Note that since there is no ``\t{Presentation Error}'' outcome in the contest rules, if you fail to follow these requirements, you will get ``\t{Wrong answer}'' outcome.

Your code will be executed by formal rules described in the statement. It must convert array \t{\$a} from the initial state to the target state.