{
    \parindent=1cm
    \begin{tabular}{l@{\extracolsep{1cm}}l}
         Идея задачи: & Виталий Аксенов\\
         Условие: & Павел Кунявский, Георгий Корнеев\\
         Проверяющая программа: & Павел Кунявский\\
         Тесты: & Павел Кунявский\\
     \end{tabular}
}

Каждый раз когда мы передвигаем блок, его имеет смысл сдвигать как можно правее. Таким образом, мы сможем двигать блоки по очереди, и каждый раз блок будет передвигаться на $b - a$. Так как общее расстояние, на которое требуется передвинуть длинный блок равно $n - b$,  то на его передвижение потребуется $2\left\lceil \frac{n - b}{b - a} \right\rceil$ действий. Так как для передвижения маленького блока потребуется еще одно действие, то общее число действий будет равно $2\left\lceil \frac{n - b}{b - a} \right\rceil + 1$.