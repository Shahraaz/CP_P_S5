The first line contains two integers $w$ and $h$, the width and the height of the golf course in inches ($4 \le w, h \le 5 \cdot 10^8$). A coordinate system is introduced in such a way that the corners of the golf course have coordinates $(0, 0)$, $(w, 0)$, $(w, h)$, $(0, h)$, and the point $(1, 1)$ is to the north-east of the point $(0, 0)$.

The second line contains a single integer $n$, denoting the number of vertices in the polygon ($4 \le n \le 1000$).

Each of the following $n$ lines contains two integers $x_i$ and $y_i$, denoting the coordinates of the $i$-th vertex ($1 \le x_i \le w - 1$; $1 \le y_i \le h - 1$). The vertices are listed in traversal order.

All polygon vertices are distinct and none of them lie at the polygon's edge.
All polygon edges are either vertical ($x_i = x_{i+1}$; $x_n=x_1$) or horizontal ($y_i = y_{i+1}$; $y_n=y_1$), and none of them intersects another.

The next line contains a single integer $t$, denoting the number of starting points for a shot ($1 \le t \le 100$).

Each of the following $t$ lines contains two integers $\tilde{x}_i$ and $\tilde{y}_i$, denoting the coordinates of the $i$-th starting point ($1 \le \tilde{x}_i \le w - 1$; $1 \le \tilde{y}_i \le h - 1$). No starting point lies inside or on the border of the pond.

Note that all starting points are independent of each other. In particular, you can assume that there is exactly one ball inside the golf course at any moment.
