{
    \parindent=1cm
    \begin{tabular}{l@{\extracolsep{1cm}}l}
         Идея задачи: & Георгий Корнеев\\
         Условие: & Геннадий Короткевич\\
         Проверяющая программа: & Геннадий Короткевич\\
         Тесты: & Геннадий Короткевич\\
     \end{tabular}
}

Мяч тонет в момент касания одной из сторон многоугольника. Переберем все стороны и найдем время первого касания для каждой из них. Тогда ответом на задачу, будет минимальное среди полученных времен.

Рассмотрим сторону $i$. Не уменьшая общности, будем считать что она вертикальная (случай горизонтальной стороны решается аналогично с точностью до замены координат). Введем обозначения $x_s=x_{i,1}=x_{i,2}$, $y_1=y_{i,1}$ и $y_2=y_{i,2}$, тогда отрезок будет иметь вид $x=x_s$ и $y_1 \le y \le y_2$.

Для удобства, вместо отражения мяча от сторон поля для гольфа, будет отражать само поле. При этом, мяч будет лететь по прямой $(\tilde{x} + t, \tilde{y} + t)$, а рассматриваемая сторона многоугольника даст бесконечную последовательность отрезков на решетке размера $2\cdot w \time 2\cdot h$. Выпишем явным образом все виды этих отрезков:
\begin{shortitems}
\item $x = 2wk + x_s$, $2hl + y_1 \le y \le 2hl + y_2$
\item $x = 2wk + x_s$, $2hl + 2h - y_2 \le y \le 2hl + 2h - y_1$
\item $x = 2wk + 2w - x_s$, $2hl + y_1 \le y \le 2hl + y_2$
\item $x = 2wk + 2w - x_s$, $2hl + 2h - y_2 \le y \le 2hl + 2h - y_1$
\end{shortitems}
где $k$ и $l$~--- целые числа.
Заметим, что каждый вид отрезков можно задать как $x=2wk+\overline{x}$, $2hl + \overline{y}_1 \le y \le 2hl + \overline{y}_2$, для некоторых $\overline{x}$ и $\overline{y}$. Решим задачу отдельно для отрезка каждого вида и выберем среди полученных времен минимальное.

Траектория мяча пересекает последовательность прямых вида $x=2wk+\overline{x}$ с периодом $2w$. Таким, образом, времена пересечения имеют вид $t_k=t_0 + 2wk$ для некоторого $0 \le t_0 < 2w$ и $k \ge 0$. Найдем решение с минимальным $k$. В начале проверим $k = 0$. Если $\overline{y}_1 \le \tilde{y} + t_0 \le 2hl \le \overline{y}_2$, то ответ найден. В противном случае, задача свелась к поиску минимального решения диофантового неравества вида $L \le ak \bmod m \le R$, где $m=2hl$, $a=2w$, $L=(\overline{y}_1 - \tilde{y}) \bmod m$, $R=(\overline{y}_2 - \tilde{y}) \bmod m$.

Решим это неравенство, рассмотрев несколько случаев

\begin{shortnums}
\item Если $2a > m$, то сведем к решению неравенства $m - R \le ((m - a)k) \bmod m \le m - L$.
\item Если $k=\left\lceil \frac{L}{a}\right\rceil$ являетсе решением, но оно минимальное.
\item Иначе, если $m$ делится на $a$, то решений не существует.
\item Иначе, рассмотрим первые значения, после каждого переполнения по модулю $m$. Каждый раз оно будет изменяться на $m \bmod a$. Нам требуется найти минимальное значение, которое удовлетворяет по модулю $a$ попадет в отрезок $L \bmod a..R \bmod a$. Так как решение не было найдено во втором случае, то $k=\left\lceil \frac{L}{a}\right\rceil$ не является решением и $L \bmod a < R \bmod a$. Таким образом, мы свели задачу к решению неравенства $(L \bmod a) \le ((a - (m \bmod a)) \cdot k' \le (R \bmod a)$. Решим его тем же алгоритмом. Пусть $k'$~--- минимальное решение нового неравенства, тогда $\left\lceil \frac{k'm + l}{a} \right\rceil$~--- минимальное решение исходного неравества. Если новое неравенство не имеет решений, то исходного неравенство так же не имеет решение.
\end{shortnums}

Временная сложность алгоритма решения неравества будет $O(\log m)$, так как после
каждого применения случая 1, $a$ уменьается как минимум в два раза. Таким образом, общая сложность полученного решения $O(tn\log(wh))$.
