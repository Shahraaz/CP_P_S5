\begin{tutorial}{Lengths and Periods}

{
    \parindent=1cm
    \begin{tabular}{l@{\extracolsep{1cm}}l}
         Идея задачи: & Антон Гардер\\
         Условие: & Антон Гардер\\
         Проверяющая программа: & Антон Гардер\\
         Тесты: & Антон Гардер\\
     \end{tabular}
}

В задаче требовалость найти подстроку, у которой длина, делённая на минимальный период, максимальна. Назовём такую подстроку оптимальной.

Если ответ не равен 1/1, то у $w$ есть подстрока с периодом, не равным её длине. Рассмотрим $w[i..j)$, пусть её наименьший период $p$, тогда $LCP(i, j - p) \ge p$, так как наибольший общий префикс у $s[i..)$ и $s[j - p..n)$ не меньше $p$. Заметим, что если $LCP(i, j - p) > p$, то наименьший период строки $w[i..j + 1)$ также не больше $p$, в то время как её длина больше длины $w[i..j)$, то есть $w[i..j)$ не может быть оптимальной подстрокой.

Теперь нас интересуют только подстроки вида $w[i : j + LCP(i,j))$. Каждая из них обновляет нам ответ значением $j - i + LCP(i, j) / (j - i)$, что равно $1 + LCP(i, j) / (j - i)$. Таким образом, из всех пар $i < j$ с равными $LCP(i, j)$ оптимальной является та, для которой $j - i$ минимально.

Построим суффиксное дерево на строке $(w + \t{\char 35})$. Пусть $len(v)$~-- длина пути от корня до $v$. LCP любых двух суффиксов, проходящих через $v$ не меньше $len(v)$.
Требуется найти среди этих суффиксов два самых близких~--- таких, у которых разница начальных позиций ($d$) минимальна, и попробовать обновить ответ значением $1 + len(v) / d$.

Для этого построим для каждой вершины $v$ упорядоченное множество $S_v$, содержащее начальные позиции всех суффиксов, проходящих через $v$. Для этого рекурсивно построим такие множества для детей $v$, обновим для них ответ, а затем сольём их. Во время слияния, когда добавляем начало суффикса в множество $S_v$, найдём его ближайших соседей и обновим наименьшую разницу позиций. После слияния всех множеств обновим ответ. Для того, чтобы слияние работало быстро, будем всегда сливать меньшее к большему.

Таким образом, общее время работы алгоритма будет $O(|w| \log^2 |w|)$.

\end{tutorial}
