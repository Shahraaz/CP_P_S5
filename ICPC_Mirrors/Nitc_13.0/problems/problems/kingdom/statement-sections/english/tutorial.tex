{
    \parindent=1cm
    \begin{tabular}{l@{\extracolsep{1cm}}l}
         Идея задачи: & Павел Маврин\\
         Условие: & Павел Маврин\\
         Проверяющая программа: & Павел Маврин\\
         Тесты: & Павел Маврин\\
     \end{tabular}
}

Заметим, что если не требуется максимизировать площадь прямоугольника `\t{A}', то задачу можно решить следующим образом. Растянем каждую букву по вертикали, пока она не упрется в другую букву или край королевства. Таким образом, получится набор полных столбцов. Растянем каждый столбец по горизонтали, пока он не упрется в соседний или край королевства. Таким образом, королевство будет разбито на прямоугольные регионы.

Научимся максимизировать прямоугольник `\t{A}'. Для этого найдем прямоугольник максимальной площади, содержащий букву `\t{A}', и не содержащий других букв. Это стандартная задача, которая решается за время $O(nm)$. Докажем, что оставшуюся часть королевства можно разбить на прямоугольники требуемым образом. Разобьем её на четыре больших прямоугольника: выше прямоугольника `\t{A}', ниже его, слева и справа. Каждый из больших прямоугольников либо пуст, либо содержит хотя бы одну букву (в противном случае, мы бы могли увеличить прямоугольника `\t{A}' в соответствующую сторону). Решив задачу без максимизации для каждого большого прямоугольника, мы разобьем все королевство на прямоугольники. Таким образом, общее время решения задачи $O(nm)$.
