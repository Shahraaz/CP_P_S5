\begin{tutorial}{Just the Last Digit}

{
    \parindent=1cm
    \begin{tabular}{l@{\extracolsep{1cm}}l}
         Идея задачи: & Артем Васильев\\
         Условие: & Артем Васильев\\
         Проверяющая программа: & Артем Васильев\\
         Тесты: & Артем Васильев\\
     \end{tabular}
}

Математическая постановка задачи имеет вид: дано число путей (взятое по модулю 10) между каждой парой вершин в ациклическом направленном графе, требуется востановить граф.

Начнем восстанавливать граф с вершины 1. Если число путей $1 \leadsto 2$ равно нулю, то ребра нет, иначе ребро есть, при этом, $1 \leadsto 2$ должно быть равно единице. Если мы удалим это ребро, то число путей $1 \leadsto i$ уменьшится на число путей $2 \leadsto i$. Переберем $i$ от $3$ до $n$ и вычтем число путей $2 \leadsto i$ из числа путей $1 \leadsto i$ (все вычисления производятся по модулю 10).

Далее рассмотрим число путей $1 \leadsto 3$. Если это ноль, то ребра $1 \to 3$ нет, а если единица, то ребро есть и мы можем вычесть число путей $3 \leadsto i$ из $1 \leadsto i$ для $i$ от $4$ до $n$. Повторим эту процедуру для $1 \leadsto 4$, $1 \leadsto 5$ и так далее. Таким образом, мы найдем все ребра из вершины $1$ за $\mathcal{O}(n^2)$.

Аналогично переберем ребра из вершин $2$, $3$, \ldots, $n$. Общее время исполения будет $\mathcal{O}(n^3)$.


\end{tutorial}
