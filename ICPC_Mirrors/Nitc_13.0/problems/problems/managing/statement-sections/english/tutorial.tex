{
    \parindent=1cm
    \begin{tabular}{l@{\extracolsep{1cm}}l}
         Идея задачи: & Михаил Мирзаянов\\
         Условие: & Михаил Мирзаянов\\
         Проверяющая программа: & Михаил Мирзаянов\\
         Тесты: & Михаил Мирзаянов\\
     \end{tabular}
}

В задаче требовалось найти количество троек $(i, j, k)$, таких что $1 \le i < j < k \le n$ и $a_k-a_j=a_j-a_i$. Заметим, что зная $a_i$ и $a_j$, можно найти требуемое значение $a_k=2a_j-a_i$.

Будем перебирать $j=n-1..2$ (по убыванию) и $i=1..j-1$, тогда задача сведется к тому, что бы посчитать количество таких $k$, что $k > j$ и $a_k=2a_j-a_i$.

Будем поддерживать ассоциативный массив $C[v]$~--- количество таких $k$, что $k > j$ и $a_k=v$. Тогда при фиксированных $i$ и $j$, ответом будет $C[2a_j-a_i]$. При переходе от $j$ к $j - 1$ в $C$ изменится только одно значение: $C[a_j]:=C[a_j]+1$.

Таким образом, используя реализацию ассоциативного массива на основе хеш-таблицы, мы сможем решить задачу за $O(1)$ при фиксированных $i$ и $j$, и $O(n^2)$ для всех пар $i$ и $j$. Если же вместо хеш-таблицы используем упорядоченный ассоциативный массив (например, \t{std::map} в C++), то задача будет решена за $O(n^2 \log n)$.
