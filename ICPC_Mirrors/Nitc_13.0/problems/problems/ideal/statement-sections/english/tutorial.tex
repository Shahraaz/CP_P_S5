{
    \parindent=1cm
    \begin{tabular}{l@{\extracolsep{1cm}}l}
         Идея задачи: & Георгий Корнеев\\
         Условие: & Илья Збань\\
         Проверяющая программа: & Илья Збань\\
         Тесты: & Илья Збань\\
     \end{tabular}
}

Минимальная пирамида, содержащая $i$-й обелиск имеет центр $(x_i, y_i)$ и высоту $h_i$. Основание такой пирамиды~--- квадрат с вершинами $(x_i-h_i, y_i-h_i)$ и $(x_i+h_i, y_i+h_i)$. При этом, основание любой пирамиды, содержащей этот обелиск должен содержать и указанный квадрат.

Таким образом, мы свели задачу к поиску на плоскости наименьшего квадрата, содержащего все квадраты, соответсвующие обелискам. Рассмотрим ограничичвающий прямоугольник этих квадратов: $x_l=\min(x_i-h_i)$, $x_r=\max(x_i+h_i)$, $y_l=\min(y_i-h_i)$, $y_r=\max(y_i+h_i)$. Тогда минимальная пирамида будет иметь высоту $h=\lceil {{\max(x_r-x_l, y_r-y_l)} \over {2}} \rceil$, а ее центр может быть расположен в точке $x={{x_l + x_r} \over 2}, y={{y_l + y_r} \over 2}$.
