{
    \parindent=1cm
    \begin{tabular}{l@{\extracolsep{1cm}}l}
         Идея задачи: & Михаил Дворкин\\
         Условие: & Михаил Дворкин\\
         Проверяющая программа: & Михаил Дворкин\\
         Тесты: & Михаил Дворкин\\
     \end{tabular}
}

Чтобы декартово дерево имело глубину $n$ достаточно, чтобы и $x$, и $y$ были монотонными.

Так как синус имеет период $2\pi$, будем искать решение вида $sin(k(2\pi+\varepsilon))$, то есть последовательность $sin(0) < sin(2\pi+\varepsilon) < sin(2(2\pi+\varepsilon)) < \cdots < sin(k(2\pi+\varepsilon))$. Для этого переберем все натуральные числа до $(2^{31} - 1) / 50\,000$ и найдем среди них число, имеющее минимальный остаток по модулю $2\pi$. Таким числом окажется $710\approx 113\cdot{}2\pi+0.00006$.

К сожалению, $0.00006 \cdot 50\,000 = 3 > \pi/2$, поэтому синусы выбранных чисел не будут возрастающей последовательностью. Однако, мы можем начать не с $0$ а c $-710 \cdot 25\,000$, тогда мы получим интервал отклонений от чисел вида $2\pi\cdot k$ от $-1.5 = -0.00006 \cdot 25\,000$ до $1.5 = 0.00006 \cdot 25\,000$, что позволяет получить серию из $50\,001$ возрастающих значений синуса.