\begin{tutorial}{Double Palindrome}

{
    \parindent=1cm
    \begin{tabular}{l@{\extracolsep{1cm}}l}
         Идея задачи: & Геннадий Короткевич\\
         Условие: & Геннадий Короткевич\\
         Проверяющая программа: & Геннадий Короткевич\\
         Тесты: & Геннадий Короткевич\\
     \end{tabular}
}

Для того, чтобы посчитать число двойных палиндромов длины $n$, посчитаем вспомогательную функцию $R(n, k)$~--- число способов построить строку длины $n$ над алфавитом размера $k$ в виде конкатенации двух палиндромов. Для этого переберём длину первого палиндрома от $0$ до $n - 1$. Так как число палиндромов длины $l$ над алфавитом размера $k$ равно $k^{\lceil l / 2\rceil}$, то получим $R(n, k) = \sum_{l=0}^{n-1} k^{\lceil l / 2\rceil} \cdot k^{\lceil (n - l) / 2\rceil}$.

Однако при таком разбиении некоторые двойные палиндромы будут посчитаны два раза, например строку ``\t{abacabacabac}'' можно представить как пару палиндромов тремя способами: ``\t{aba$|$cabacabac}'', ``\t{abacaba$|$cabac}'' и ``\t{abacabacaba$|$c}''. Таким образом, мы посчитаем эту строку три раза.

Заметим, что строка $s$ длины $n$ является двойным палиндромом тогда и только тогда, когда $s$ равна циклическому сдвигу развернутой строки $s$, т.~е. существует число $k$ (которое на самом деле равно длине первого из палиндромов, на которые разбивается $s$) такое, что $\forall i: s_i = s_{k - 1 - i \mod n}$.

Пусть для строки $s$ нашлось два способа разбить её на два палиндрома, при этом длины первого палиндрома в этих способах равны $l_1$ и $l_2$ соответственно ($l_1 < l_2$). Тогда $$s_i = s_{l_1 - 1 - i \mod n} = s_{l_2 - l_1 + i \mod n},$$ то есть строка $s$ периодична с периодом $l_2 - l_1$, и даже с периодом $\gcd(n, l_2 - l_1)$. Пусть $p$~--- минимальный период двойного палиндрома $s$, тогда $p$~--- тоже двойной палиндром, и $p$ представляется в виде конкатенации не более двух палиндромов единственным способом, при этом в $R(n, k)$ мы посчитали $s$ ровно $\frac{|s|}{|p|}$ раз.

Давайте посчитаем значение $D(n, k)$~--- количество двойных палиндромов длины $n$ над алфавитом размера $k$, которые разбиваются на два палиндрома единственным способом. Для $D(n, k)$ верна следующая рекуррентная формула:

$$ D(n, k) = R(n, k) - \sum_{\substack{l|n\\l<n}} \frac{n}{l} D(l, k) $$

Тогда ответом будет $\sum\limits_{n} \sum\limits_{l|n} D(l, k)$. Суммарное время работы составляет $\mathcal{O}(n \log n)$.


\end{tutorial}
