\problemname{Disposable Switches}

\illustration{0.42}{cables_small}{\href{https://flic.kr/p/7AMDgK}{Picture} by Ted Sakshaug on Flickr, cc by}%
\noindent
Having recently been hired by Netwerc Industries as a network engineer, your
first task is to assess their large and dated office network. After mapping out
the entire network, which consists of network switches and cables
between them, you get a hunch that, not only are some of the switches
redundant, some of them are not used at all! Before presenting this to
your boss, you decide to make a program to test your claims.

The data you have gathered consists of the map of the network, as well as the
length of each cable. While you do not know the exact time it takes to send a
network packet across each cable, you know that it can be calculated as
$\ell / v + c$, where
\begin{itemize}
    \item $\ell$ is the length of the cable,
    \item $v$ is the propagation speed of the cable, and
    \item $c$ is the overhead of transmitting the packet on and off the cable.
\end{itemize}
You have not been able to measure $v$ and $c$.  The only thing you know about
them is that they are real numbers satisfying $v > 0$ and $c \ge 0$, and that
they are the same
for all cables. You also know that when a network packet is being transmitted
from one switch to another, the network routing algorithms will ensure that the
packet takes an \emph{optimal path}---a path that minimises the total transit
time.

Given the map of the network and the length of each cable, determine which
switches could never possibly be part of an optimal path when transmitting a network packet from
switch $1$ to switch $n$, no matter what the values of $v$ and $c$ are.

\section*{Input}
The input consists of:
\begin{itemize}
    \item One line with two integers $n$, $m$ ($2 \leq n \leq 2\,000$, $1 \leq m \leq 10^4$), the number
    of switches and the number of cables in the network. The switches are numbered
    from $1$ to $n$.
    \item $m$ lines, each with three integers $a$, $b$ and $\ell$ ($1 \leq a, b
    \leq n$, $1\leq \ell \leq 10^9$), representing a network cable of length
    $\ell$ connecting switches $a$ and $b$.
\end{itemize}

There is at most one cable connecting a given pair of switches, and no cable
connects a switch to itself. It is guaranteed that there exists a path between
every pair of switches.

\section*{Output}
First output a line with an integer $k$, the number of switches that could never possibly be
part of an optimal path when a packet is transmitted from switch $1$ to switch $n$.
Then output a line with $k$ integers,
the indices of the $k$ unused switches, in increasing order.

